\section{Data}
\label{sec:data}
% Description of the dataset used for analysis
We use Python \citep{python3}, Pandas \citep{reback2020pandas}, NumPy \citep{harris2020array}, Statsmodels \citep{seabold2010statsmodels} to analyze S\&P 500 price data from Finaeon's Global Financial Dataset \citep{finaeon} and the Fama-French 3-factor data library \citep{french_website}. 
We use Matplotlib for plotting \citep{Hunter2007}.
We get the S\&P500's current constituents from the Finaeon website, and download a time-series of daily prices for each stock using Finaeon's API. 
Kenneth french's website \citep{french_website} provides the risk-free rate, market excess return, and daily returns of the Size and Value portfolios. 
The risk-free rate they provide is the 1-month T-bill rate, and the market risk premium is the difference between the total US stock market's return and the risk-free rate.
The size and value portfolios are constructed according to \citet{fama_french_1993}.
By only using the return histories of the current S\&P 500 constituents, we introduce a survivorship bias in our analysis. Any risk premia we include will be biased, as we only consider firms that have survived to the present day, and are large.
The purpose of this article is to introduce concepts of modern portfolio theory, and not to provide a comprehensive analysis of risk factors and premia, so we
note this deficiency and move forward with our analysis.

\begin{table}
    \centering
    \begin{tabular}{|c|c|c|}
        \hline
        \textbf{Company Name} & \textbf{Ticker}\\
        \hline
        Apple Inc. & AAPL\\
        General Electric Co. & GE\\
        International Business Machines Corp. & IBM\\
        The Coca-Cola Company & KO\\
        Microsoft Corp. & MSFT\\
        \hline
    \end{tabular}
    \caption{Sample assets and their full names}
    \label{tab:sample_assets_w_names}
\end{table}

Throughout this article, we'll use a few sample assets to illustrate concepts. We'll refer to them by ticker symbol, and their full names are shown in table \ref{tab:sample_assets_w_names}.
These are randomly chosen from the S\&P 500, and are not meant to be representative of the index as a whole.

\begin{table}[ht]
    \centering
    \vspace{1em}
    \begin{tabular}{llrrrrrrr}
\toprule
date & symbol & series\_id & close & return & RF & Mkt-RF & SMB & HML \\
\midrule
2015-01-02 00:00:00 & CSCO & 58375 & 27.61 & -0.01 & 0.00 & -0.00 & -0.01 & 0.00 \\
2015-01-05 00:00:00 & CSCO & 58375 & 27.06 & -0.02 & 0.00 & -0.02 & 0.00 & -0.01 \\
2015-01-06 00:00:00 & CSCO & 58375 & 27.05 & -0.00 & 0.00 & -0.01 & -0.01 & -0.00 \\
2015-01-07 00:00:00 & CSCO & 58375 & 27.30 & 0.01 & 0.00 & 0.01 & 0.00 & -0.01 \\
2015-01-08 00:00:00 & CSCO & 58375 & 27.51 & 0.01 & 0.00 & 0.02 & -0.00 & -0.00 \\
\bottomrule
\end{tabular}

    \caption{Sample of the dataframe used for analysis, RF, MKT, SMB, HML are the daily risk-free rate, market excess return, size and value portfolio returns respectively.}
    \vspace{1em}
\end{table}

We clean the data by removing outliers, windsorizing the returns to be within 3 standard deviations of the mean. This helps ensure that our analyses are not influenced by extreme market movements which may be the result of data errors or unusual events.