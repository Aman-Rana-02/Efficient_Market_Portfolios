\section{Introduction}
\label{sec:introduction}
Modern portfolio theory is built on the premise that investors can optimize their returns by balancing systematic risk exposures.
Over time, several frameworks have emerged to guide this optimization.
The Capital Asset Pricing Model (CAPM) came first, based on the belief that market risk drove expected returns, 
suggesting that firm-specific risks can be diversified away and markets priced all risk. 
Markowitz Mean-Variance Optimization (MVO) builds on this notion of diversification,
we demonstrate how optimal portfolios can be created to minimize risk for a targeted return.
Arbitrage Pricing Theory (APT) extends beyond a single factor by incorporating multiple risk factors (such as size and value)
to more accurately capture the sources of systematic risk.

In this article, we use historical price data for S\&P 500 constituents gathered from Finaeon's Global Financial Dataset, and offer code and notes.
First, we describe how the data is assembled and highlight potential biases such as survivorship bias.
Next, we delve into the CAPM, showing how betas are estimated and how they relate to expected returns.
We then explore deriving an efficient frontier of optimal portfolios.
Finally, we discuss the APT approach, including how Fama-Macbeth regressions help estimate multiple factor betas and risk premia.
We provide a brief overview of how these models shape efficient portfolio construction.

The remainder of the article is structured as follows:
\begin{itemize}
    \item Section~\ref{sec:data} provides an overview of the dataset used for analysis.
    \item Section~\ref{sec:capital_asset_pricing_model} explores the Capital Asset Pricing Model (CAPM) and its implications.
    \item Section~\ref{sec:diversification_and_portfolios} discusses the importance of diversification and the construction of efficient portfolios.
    \item Section~\ref{sec:arbitrage_pricing_theory} touches on Arbitrage Pricing Theory (APT) and its applications.
    \item Section~\ref{sec:conclusion} concludes by discussing limitations of the methods discussed.
\end{itemize}