\section{Conclusion}
\label{sec:conclusion}
\subsection{Limitations}
In this article and the provided repository we briefly introduce some principles of modern portfolio theory.
We showed how APT portfolios could be used to add more informative risks, and provided code with Fama-Macbeth regressions to facilitate 
future work on exploring different risk factors. This framework also enables factor neutral portfolios, which can be used to create portfolios that are neutral to certain risk factors, but
the code provided does not implement this.

The limitations of these models remain that they use historical data, APT attempts to address this by using expected returns generated by 
risk premia, risk premia exposures, and risk factor covariances, which may be more stable than individual asset returns. The out of sample performance
of MVO portfolios is a common criticism, and frameworks such as Hierarchical Risk Parity (HRP) and Black-Litterman have been proposed to address this.

Additionally, in practice there are even more factors added, and even factor covariance matrices can start to suffer from the numerical instability that vanilla MVO suffers from.
The code provided in this repository is a starting point for exploring these models, and we encourage readers to explore the literature and
implementations of these models to gain a deeper understanding of their limitations and potential improvements.
